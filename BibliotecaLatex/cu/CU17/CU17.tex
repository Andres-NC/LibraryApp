%%caso de uso 17 creacion de roles
\begin{UseCase}{CU17}{Crear nuevo rol}{
		El administrador da de alta un nuevo rol para poder ser asignado a uno o varios trabajadores.
}
	\UCitem{Versión}{1.1}
	\UCitem{Actor}{Administrador.}
	\UCitem{Propósito}{El usuario puede agregar nuevos roles al sistema para que cada trabajador tenga asignado su rol específico.}
	\UCitem{Entradas}{Nombre de Rol}
	\UCitem{Salidas}{Se crea un nuevo registro de rol en la base de datos del sistema y se muestra el mensaje \MSGref{MSG01}{Operación exitosa}.}
	\UCitem{Precondiciones}{Existe el usuario Administrador dado de alta en la base de datos.}
	\UCitem{Postcondiciones}{El sistema redirige al usuario a la pantalla gestionar roles y se genera un nuevo registro en la base de datos.}
	\UCitem{Autores}{Guarneros Santana Víctor Hugo y Mújica Márquez Víctor Edgar.}
	\UCitem{Status}{En revisión}
	\UCitem{Responsable de revisión}{Esteban Martínez}
\end{UseCase}


%%trayectoria principal
\begin{UCtrayectoria}{Crear rol}
	\UCpaso[\UCactor] El Administrador selecciona el botón \IUbutton{Crear} de la pantalla \IUref{IU02}{Directorio de Roles}
	\UCpaso[\UCsist] Despliega la pantalla \IUref{IU17}{Crear rol}
	\UCpaso[\UCactor] El administrador asigna el nombre al nuevo rol
	\UCpaso[\UCactor] Selecciona el botón \IUbutton{Crear} de la pantalla \IUref{IU17}{Crear rol}
	\UCpaso[\UCsist] Verifica regla de negocio \BRref{RN01}{Duplicidad de roles} y crea el nuevo rol, \Trayref{A}
	\UCpaso[\UCsist] Muestra mensaje %\MSGref{MSG1}{Operación exitosa}
	Operación exitosa
	\UCpaso[\UCsist] Regresa a la pantalla \IUref{IU02}{Directorio de Roles}
\end{UCtrayectoria}


\begin{UCtrayectoriaA}{A}{Registro ya existente}
	\UCpaso[\UCsist] Muestra mensaje \MSGref{MSG03}{Registro ya existente}.
	\UCpaso[\UCactor] Ingresa un nuevo valor valido.
	\UCpaso[\UCsist] Verifica regla de negocio \BRref{RN01}{Duplicidad de roles} y crea el nuevo rol, \Trayref{A}
	\UCpaso[\UCsist] Muestra mensaje \MSGref{MSG01}{Operación exitosa}
	\UCpaso[\UCsist] Regresa a la pantalla \IUref{IU01}{Directorio de Roles}
\end{UCtrayectoriaA}