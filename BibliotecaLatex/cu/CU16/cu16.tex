% \IUref{IUAdmSucur}{Administrador}
% \IUref{IUConsDA}{Registrar empleado}
% Version 1


% Copie este bloque por cada caso de uso:
%-------------------------------------- COMIENZA descripción del caso de uso.

%\begin{UseCase}[archivo de imágen]{UCX}{Nombre del Caso de uso}{
	\begin{UseCase}{CU16}{Registrar empleado}{
		El administrador podrá registrar en el sistema a un nuevo empleado, con la finalidad de tener guardada la información de los trabajadores de la biblioteca para consulta y control de la misma. Lo hará introduciendo los datos del nuevo empleado: nombre completo, RCF, CURP, teléfono, email, dirección, el tipo de empleado y la duración de su contrato.
	}
		\UCitem{Versión}{2.0}
		\UCitem{Estado}{En revisión}
		\UCitem{Actor}{Administrador.}
		\UCitem{Propósito}{Almacenar en la base de datos los datos de todos los empleados de la biblioteca.}
		\UCitem{Entradas}{Nombre, apellido paterno, apellido materno, CURP, RFC, telefono, email, tipo de empleado, duracion del contrato y dirección}
		\UCitem{Salidas}{Registro en la base de datos junto con el mensaje \MSGref{MASG1}{Operación realizada exitosamente.}, mensaje \MSGref{E3}{Datos incompletos. Favor de completar todos los campos de entrada.}, mensaje \MSGref{MASG6}{Confirma que los datos ingresados sean correctos.}, mensaje \MSGref{E5}{Registro ya existente en el catalogo.} y mensaje \MSGref{E1}{Error al conectar a la BD.}}
		\UCitem{Precondiciones}{
			\begin{itemize}
				\item El empleado no debe de estar registrado.
				\item El empleado debe de contar con su RFC.
			\end{itemize}
			}
		\UCitem{Postcondiciones}{Nuevo empleado registrado en el sistema.}
		\UCitem{Tipo}{Caso de uso primario.}
		\UCitem{Autor}{Durán Martínez Josué y Miguel Ángel Castañeda Sánchez.}
		\UCitem{Revisor}{Arturo Rodriguez Cervantes y Monserrat Cerón Rodriguez.}
	\end{UseCase}
	
	
%%------------Trayectoria Principal-----------------%%		
	
	\begin{UCtrayectoria}{Principal}
	
		\UCpaso[\UCactor] Ingresa a la \IUref{UI16.1}{Pantalla Registrar empleado} dando clik en el botón \IUbutton{Registrar empleado}. 
		
		\UCpaso Despliega la \IUref{UI16.1}{Pantalla Registrar empleado}.

		\UCpaso[\UCactor] Llena todos los campos que son: 
			\begin{itemize}
				\item Nombre.
				\item Apellido paterno.
				\item Apellido materno.
				\item RFC.
				\item CURP.
				\item Telefono.
				\item Email.
				\item Tipo de empleado.
				\item Duración del contrato.
				\item Dirección.
			\end{itemize} 
		\label{CU16RegistrarEmpleado}

		\UCpaso[\UCactor] Da clik en el botón \IUbutton{Registrar}.\label{CU16ConectarBaseDatos}

		\UCpaso Se conecta a la base de datos. \Trayref{D}.

		\UCpaso Verifica los campos de los datos con base a la \BRref{RN36}{Ningún campo nulo.} \Trayref{A}.

		\UCpaso Valida que los datos ingresados cumplan los formatos con base a las Reglas de Negocio: 
				
			\begin{itemize}
				\item	\BRref{RN37}{Formato del nombre} 
				\item	\BRref{RN38}{Formato del RFC} 
				\item	\BRref{RN39}{Formato del CURP} 
				\item	\BRref{RN40}{Formato del telefono} 
				\item	\BRref{RN45}{Formato del email} 
				\item	\BRref{RN41}{Formato de la dirección} 
				\item	\BRref{RN44}{Formato del contrato} 
			\end{itemize} 
		\Trayref{B}.

		\UCpaso Valida que el empleado no halla sido registrado previamente con base a \BRref{RN35}{Empleado registrado} \Trayref{C}.

		\UCpaso El sistema guardara un registro en la bitacora en donde almacenara la fecha del servidor y guardara  al nuevo empleado.

		\UCpaso Muestra el Mensaje \MSGref{MASG1}{Operación realizada exitosamente.}
		

	\end{UCtrayectoria}



%%------------Trayectorias Alternativas-----------------%%	

%%------------Trayectoria A-----------------%%
		
		
		\begin{UCtrayectoriaA}{A.}{El Administrador no ha llenado todos los campos}
			\UCpaso Muestra el Mensaje \MSGref{E3}{Datos incompletos. Favor de completar todos los campos de entrada.}
			\UCpaso[\UCactor] Oprime el botón \IUbutton{Aceptar}.
			\UCpaso Continua en el paso \ref{CU16RegistrarEmpleado} del \UCref{CU16}.
			
		\end{UCtrayectoriaA}


%%------------Trayectoria B-----------------%%
		\begin{UCtrayectoriaA}{B.}{El Administrador ha ingresado los datos de manera incorrecta}

			\UCpaso Muestra el Mensaje \MSGref{MASG6}{Confirma que los datos ingresados sean correctos.}
			\UCpaso[\UCactor] Oprime el botón \IUbutton{Aceptar}.
			\UCpaso Continua en el paso \ref{CU16RegistrarEmpleado} del \UCref{CU16}.

		\end{UCtrayectoriaA}

%%------------Trayectoria C-----------------%%
		\begin{UCtrayectoriaA}{C.}{El empleado ya esta registrado}

			\UCpaso Muestra el Mensaje \MSGref{E5}{Registro ya existente en el catalogo.}
			\UCpaso[\UCactor] Oprime el botón \IUbutton{Aceptar}.
			\UCpaso Continua en el paso \ref{CU16RegistrarEmpleado} del \UCref{CU16}.

		\end{UCtrayectoriaA}

%%------------Trayectoria D-----------------%%
		\begin{UCtrayectoriaA}{D.}{Error al conectar a la base de datos}

			\UCpaso Muestra el Mensaje \MSGref{E1}{Error al conectar a la BD.}
			\UCpaso[\UCactor] Oprime el botón \IUbutton{Aceptar}.
			\UCpaso Continua en el paso \ref{CU16ConectarBaseDatos} del \UCref{CU16}.

		\end{UCtrayectoriaA}		

%-------------------------------------- TERMINA descripción del caso de uso.