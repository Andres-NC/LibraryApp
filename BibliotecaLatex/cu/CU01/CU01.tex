%%caso de uso 01 gestión de roles
\begin{UseCase}{CU01}{Gestión de roles}{
	Mostrar al administrador las opciones de crear, modificar, consultar y eliminar rol para que seleccione la acción que desea ejecutar y redirigir a la pantalla correspondiente.
}
	\UCitem{Versión}{1.1}
	\UCitem{Actor}{Administrador.}
	\UCitem{Propósito}{El usuario puede administar los roles que se requieren para los empleados de la biblioteca.}
	\UCitem{Entradas}{Gestionar Roles}
	\UCitem{Salidas}{Pantalla correspondiente a la acción de gestión deseada.}
	\UCitem{Precondiciones}{Existe el usuario Administrador dado de alta en la base de datos.}
	\UCitem{Postcondiciones}{El sistema muestra la pantalla que corresponde a la acción de gestión que el administrador desea.}
	\UCitem{Autores}{Guarneros Santana Víctor Hugo y Mújica Márquez Víctor Edgar.}
	\UCitem{Status}{En revisión}
	\UCitem{Responsable de revisión}{Esteban Martínez}
\end{UseCase}


%%trayectoria principal

\begin{UCtrayectoria}{Principal}
	%paso1: actor
	\UCpaso[\UCactor] El Administrador selecciona el botón \IUbutton{Gestionar Roles} de la pantalla \IUref{IU01}{Pantalla principal}
	%paso2: sistema
	\UCpaso[\UCsist] Despliega la pantalla \IUref{IU02}{Directorio de Roles}
	%%paso 3: el usuario elije la acción
	\UCpaso[\UCactor] El administrador selecciona una opción de gestión de roles dando clic en el botón correspondiente a cada acción: \IUbutton{Crear} \IUbutton{Modificar} \IUbutton{Eliminar} \IUbutton{Consultar} .
	%%paso 4
	\UCpaso[\UCsist] Redirecciona al usuario a la pantalla correspondiente a cada botón.  \UCref{CU02} \UCref{CU02} \UCref{CU02} \UCref{CU02}
\end{UCtrayectoria}

%	\UCpaso[\UCactor] Selecciona el botón \IUbutton{Crear} de la pantalla \IUref{IU02}{Crear rol}
%	\UCpaso[\UCsist] Verifica regla de negocio \BRref{RN01}{Duplicidad de roles} y crea el nuevo rol, \Trayref{A}
%	\UCpaso[\UCsist] Muestra mensaje %\MSGref{MSG1}{Operación exitosa}
%	Operación exitosa
%	\UCpaso[\UCsist] Regresa a la pantalla \IUref{IU01}{Pantalla principal}