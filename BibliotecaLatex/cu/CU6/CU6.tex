% Copie este bloque por cada caso de uso:
%-------------------------------------- COMIENZA descripción del caso de uso.

	\begin{UseCase}{CU06}{Registrar Estado de Libros}{
		El administrador puede asociar diferentes estados a un libro, esto con base a una fecha de revisión que el administrador registra. Estos estados se definen en los procedimientos de devolución y préstamo. 
	}
		\UCitem{Versión}{0.1}
		\UCitem{Actor}{Administrador bibliotecario}
		\UCitem{Propósito}{El administrador pueda controlar y gestionar el préstamo de libros con base a la disponibilidad  y estado de cada uno de los libros registrados en la biblioteca.}
		\UCitem{Entradas}{El identificador del libro , la fecha -hora cuando se realiza el registro y los estados que se asocian al libro.}
		\UCitem{Salidas}{Se registra la fecha y hora del cambio de estado del libro , esto indica la disponibilidad y características durante ese cambio.}
		\UCitem{Precondiciones}{ 
			\begin{itemize}
				\item El administrador realizo el proceso CU9 de entrega de libro previamente.
				\item administrador realizo el proceso de préstamo \UCref(CU07) previamente.
				\item Un libro debe estar registrado
				\item La disponibilidad debe estar registrada
				\item El estado físico debe estar registrado.

			\end{itemize}
		}
		\UCitem{Postcondiciones}{El cambio de estado queda registrado con una fecha y hora, una disponibilidad y  0 o más condiciones físicas.}
		\UCitem{Autor}{BAUTISTA ROSALES MAURICIO}
	\end{UseCase}
		%-------------------------------------- COMIENZA descripción Trayectoria Principal
	\begin{UCtrayectoria}{Principal}

			\UCpaso[\UCactor] El administrador  realiza el proceso de entrega o préstamo \UCref(CU07) , \UCref(CU09) \Trayref{A}
			\UCpaso[\UCsist] El sistema conserva el identificador del libro e identifica si se trata de una entrega o un préstamo.
			\UCpaso[\UCsist] El sistema establece la disponibilidad con base a las reglas de negocio \BRref{RN30} \BRref{RN31}{Disponibilidad de un Libro} \Trayref{B}
			\UCpaso[\UCsist] El sistema despliega \IUref{IU6}{Pantalla Registro} solicitando al usuario que seleccione las  condiciones físicas del libro que aplique.
			\UCpaso[\UCactor] El administrador presiona el botón \IUbutton{CONTINUAR}
			\UCpaso[\UCsist] El sistema obtiene la fecha y hora en  que se hace el registro.
			\UCpaso[\UCsist] El sistema solicita a través del \MSGref{MSG6}  que el administrador confirme los datos.\Trayref{C}
			\UCpaso[\UCactor] El administrador confirma los datos presionando el botón  \IUbutton{OK}
			\UCpaso[\UCsist] El sistema registra la  disponibilidad del libro y sus condiciones físicas en la fecha y hora indicada.\Trayref{D}
			\UCpaso[\UCsist] El sistema muestra el mensaje \MSGref{MSG1} como resultado del éxito de registró.
			\UCpaso[\UCactor] El administrador confirma presionando el botón  \IUbutton{OK}

	\end{UCtrayectoria}
			%-------------------------------------- COMIENZA descripción Trayectoria Alternativa.
		\begin{UCtrayectoriaA}{A}{El administrador desea cambiar el estado de un libro en concreto.}
			\UCpaso[\UCsist]  El sistema solicita el identificador del libro.
			\UCpaso[\UCactor] El administrador ingresa el identificador del libro 
			\UCpaso[\UCsist] El sistema comprueba que exista el libro en los registros.\Trayref{F}
			\UCpaso[\UCsist] El sistema despliega la \IUref{IU6}{Pantalla Registro} y solicita al administrador que seleccione la disponibilidad del libro.
			\UCpaso[\UCactor] El administrador indica la disponibilidad del libro.
			     Trayectoria principal paso 4 \Trayref{Principal}
		\end{UCtrayectoriaA}

		\begin{UCtrayectoriaA}{B}{El administrador desea modificar la disponibilidad del libro.}
			\UCpaso[\UCactor] El administrados presiona el botón \IUbutton{MODIFICAR}
			\UCpaso[\UCsist] El sistema habilita el selector que indica la disponibilidad del libro
			\UCpaso[\UCactor] El administrador selecciona la disponibilidad correcta.
			 trayectoria principal paso 4 \Trayref{Principal}
		\end{UCtrayectoriaA}

		\begin{UCtrayectoriaA}{C}{El administrador presiona el botón \IUbutton{NO} no confirmando los datos.}
			\UCpaso[\UCsist] El sistema nuestra los datos previamente indicados por el administrador.
			\UCpaso[\UCactor] El administrador realiza las modificaciones correspondientes 
			Paso 5 trayectoria principal \Trayref{Principal}
		\end{UCtrayectoriaA}

		\begin{UCtrayectoriaA}{D}{No se logró registrar los datos }
			\UCpaso[\UCsist] El sistema verifica el número de intentos con base a la regla del negocio \BRref{RN32}\Trayref{F}
			\UCpaso[\UCsist] El sistema muestra el mensaje \MSGref{E7}.

			Trayectoria principal paso 9\Trayref{Principal}
		\end{UCtrayectoriaA}
	
		\begin{UCtrayectoriaA}{E}{Excede el Número de Intentos}
			\UCpaso[\UCsist] El sistema muestra el mensaje \MSGref{MSG7}
			\UCpaso[\UCactor]El administrador presiona el botón \IUbutton{OK} 
			Fin del caso de uso 
		\end{UCtrayectoriaA}

		\begin{UCtrayectoriaA}{F}{No existe registro del libro  }
			\UCpaso[\UCsist] El sistema muestra el mensaje \MSGref{E4}
			\UCpaso[\UCactor]El administrador presiona el botón \IUbutton{OK} 

			Trayectoria Alternativa A paso 1  \Trayref{A}
		\end{UCtrayectoriaA}
				
%-------------------------------------- TERMINA descripción del caso de uso.
