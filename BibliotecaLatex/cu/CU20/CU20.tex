%%caso de uso 20 Eliminar rol 
\begin{UseCase}{CU20}{Eliminar rol}{
		El administrador podrá eliminar el rol deseado de entre los que existen en la base de datos de acuerdo a los cargos de los trabajadores.
}
	\UCitem{Versión}{1.0}
	\UCitem{Actor}{Administrador.}
	\UCitem{Propósito}{El usuario puede eliminar los roles existentes en el sistema ingresando el nombre del rol que desea eliminar.}
	\UCitem{Entradas}{Eliminar, nombre del rol}
	\UCitem{Salidas}{El sistema redirecciona al usuario a la pantalla Directorio de Roles.}
	\UCitem{Precondiciones}{Existe el usuario Administrador dado de alta en la base de datos y al menos un registro más de otro rol.}
	\UCitem{Postcondiciones}{Un registro es eliminado en la tabla Rol de la base de datos del sistema.}
	\UCitem{Autores}{Guarneros Santana Víctor Hugo y Mújica Márquez Víctor Edgar.}
	\UCitem{Status}{En revisión}
	\UCitem{Responsable de revisión}{Esteban Martínez}
\end{UseCase}


%%trayectoria principal
\begin{UCtrayectoria}{Eliminar rol}
	\UCpaso[\UCactor] El Administrador selecciona el botón \IUbutton{Eliminar} de la pantalla \IUref{IU02}{Directorio de Roles}
	\UCpaso[\UCsist] Despliega la pantalla \IUref{IU20}{Eliminar rol}
	\UCpaso[\UCactor] El administrador ingresa el nombre del rol que desea eliminar.
	\UCpaso[\UCactor] Selecciona el botón \IUbutton{Eliminar} de la pantalla \IUref{IU20}{Eliminar rol}
	\UCpaso[\UCsist]Verifica si el rol a eliminar existe si no muestra el mensaje 
	%\MSGref{MSG02}{Registro inexistente}
	Registro inexistente
	\UCpaso[\UCsist] Muestra mensaje %\MSGref{MSG01}{Operación exitosa}
	Operación exitosa
	\UCpaso[\UCsist] Regresa a la pantalla \IUref{IU02}{Directorio de Roles}
\end{UCtrayectoria}