\begin{UseCase}{CU04}{Registrar libro}{
		Este caso de uso tiene como objetivo registar la información de un libro acorde a la norma MARC 21 para catalogar los libros.
	}
		\UCitem{Versión}{1.0}
		\UCitem{Actor}{Administrador}
		\UCitem{Propósito}{Registrar los libros entrantes conforme a MARC 21 y guardarlos en una base de datos.}
		\UCitem{Entradas}{
			\begin{itemize}
				\item Libro fisico.
			\end{itemize}					
		}
		\UCitem{Salidas}{
			\begin{itemize}
				\item Registro en la base de datos.
			\end{itemize}					
		}
		\UCitem{Precondiciones}{ 
			\begin{itemize}
				\item El Administrador debe de estar registrado. 
				\item El sistema debe de estar conectado a la base de datos.
			\end{itemize}
		}
		\UCitem{Postcondiciones}{
			\begin{itemize}
				\item Se registrará la informacion de un libro en al base de datos.
			\end{itemize}					
		}
		\UCitem{Autor}{
				Manriquez Diego
				Salas Abiran					
		}
\end{UseCase}
\begin{UCtrayectoria}{Principal}
	\UCpaso[\UCactor] El administrador seleccionara el boton \IUbotton{Registrar}	
	\UCpaso[\UCactor] El administrador introduce la informacion del libro en los campos de acuerdo a las regla de negocio RN-16. \IUref{IU04}\Trayref{A}
	\UCpaso[\UCsist] El sistema lee la informacion y compara el ISBN en la base de datos de acuerdo a la regla de negocio RN-23. \Trayref{B}
	\UCpaso[\UCsist] El sistema guarda la informacion en la base de datos. \Trayref{C}
	\UCpaso[\UCsist] Se mostrará el mensaje MSG1.
\end{UCtrayectoria}
\begin{UCtrayectoriaA}{A}{Se introdujo un campo vacio}	
			\UCpaso[\UCsist] El sistema mostrará el mensaje [E2].
			\UCpaso[\UCsist] Se reiniciará el CU, volviendo al paso 1 de la trayectoria principal. 
\end{UCtrayectoriaA}
\begin{UCtrayectoriaA}{B}{Se introdujo un ISBN existente con una editorial diferente}	
			\UCpaso[\UCsist] El sistema mostrará el mensaje [MSG8].
			\UCpaso[\UCsist] Se reiniciará el CU, volviendo al paso 1 de la trayectoria principal. 
\end{UCtrayectoriaA}
\begin{UCtrayectoriaA}{C}{Se presentó un error en la conexión y/o en la transacción actual con base de datos}
			\UCpaso[\UCsist] El sistema mostrará el mensaje [E1].
			\UCpaso[\UCsist] Se reiniciará el CU, volviendo al paso 1 de la trayectoria principal.
\end{UCtrayectoriaA}
